\documentclass[18pt,a4paper,frenchb]{article}
\usepackage[utf8]{inputenc}
\usepackage[T1]{fontenc}
\usepackage{babel}
\setlength{\parskip}{1em}

\begin{document}
\section*{Le Poète assassiné}
\subsection*{I \\ Renommée}

La gloire de Croniamantal est aujourd'hui universelle. Cent vingt-trois villes dans sept pays sur quatre continents se disputent l'honneur d'avoir vu naître ce héros insigne. J'essayerai plus loin d'élucider cette importante question.

Tous ces peuples ont plus ou moins modifié le nom sonore de Croniamantal. Les Arabes, les Turcs et autres peuples qui lisent de droite à gauche n'ont pas manqué de le prononcer Latnamaïnorc, mais les Turcs l'appellent bizarrement Pata, ce qui signifie oie ou organe viril, à volonté. Les Russes le surnomment Viperdoc, c'est-à-dire né d'un pet ; on verra plus loin la raison de ce sobriquet. Les Scandinaves, ou du moins les Dalécarliens, l'appellent volontiers \textit{quoniam}, en latin, qui signifie \textit{parce que}, mais désigne souvent les parties nobles dans les récits populaires du moyen âge. On voit que les Saxons et les Turcs manifestent à l'égard de Croniamantal le même sentiment en lui appliquant des surnoms identiques, mais dont l'origine est encore mal expliquée. On suppose que c'est une allusion euphémique à ce qui se trouvait dans le rapport médical du médecin marseillais Ratiboul sur la mort de Croniamantal. D'après cette pièce officielle, tous les organes de Croniamantal étaient sains et le médecin légiste ajoutait en latin, comme fit l'aide-major Henry pour Napoléon : \textit{partes viriles exiguitatis insignis, sicut pueri}.

Au demeurant, il est des pays où la notion de la virilité croniamantalesque a complètement disparu. C'est ainsi qu'en Moriane les nègres le nomment Tsatsa ou Dzadza ou Rsoussour, noms féminins, car ils ont féminisé Croniamantal comme les Byzantins ont féminisé le vendredi saint en en faisant sainte Parascève.

\subsection*{II \\ Procréation}

A deux lieues de Spa, sur la route bordée d'arbres tordus et de buissons, Viersélin Tigoboth, musicien ambulant qui arrivait à pied de Liége, battait le briquet pour allumer sa pipe. Une voix de femme cria :\\« Eh ! monsieur ! »

Il leva la tête et un rire éperdu éclata :\\« Hahaha ! Hohoho ! Hihihi ! tes paupières ont la couleur des lentilles d'Egypte ! Je m'appelle Macarée. Je veux un matou. »

Viersélin Tigoboth aperçut sur le bord de la route une jeune femme brune, formée de jolis globes. Qu'elle était gracieuse en jupe courte de cycliste ! Et tenant d'une main son vélo, tandis

\end{document}